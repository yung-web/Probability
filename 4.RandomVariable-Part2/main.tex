

\def\slidemode{%
  \documentclass[fleqn,aspectratio=169]{beamer}
\usepackage{pgfpages}
}
\def\handoutmode{%
  \documentclass[handout,fleqn,aspectratio=169]{beamer}
\usepackage{pgfpages}
\pgfpagesuselayout{resize to}[a4paper,landscape,border shrink=5mm]
}


%\def\pdfmode{handoutmode}
\def\pdfmode{slidemode}

\csname\pdfmode\endcsname

\mode<presentation>
{
  \usetheme{default}
  \usecolortheme{default}
  \usefonttheme{default}
  \setbeamertemplate{navigation symbols}{}
  \setbeamertemplate{caption}[numbered]
  \setbeamertemplate{footline}[frame number]  % or "page number"
  \setbeamercolor{frametitle}{fg=white}
  \setbeamercolor{footline}{fg=black}
} 
\mode<presentation>
{
  \usetheme{default}
  \usecolortheme{default}
  \usefonttheme{default}
  \setbeamertemplate{navigation symbols}{}
  \setbeamertemplate{caption}[numbered]
  \setbeamertemplate{footline}[frame number]  % or "page number"
  \setbeamercolor{frametitle}{fg=white}
  \setbeamercolor{footline}{fg=black}
} 


\usepackage[english]{babel}
%\usepackage[utf8x]{inputenc}
\usepackage{tikz}
\usepackage{courier}
\usepackage{array}
\usepackage{bold-extra}
%\usepackage{minted}
%\usepackage[thicklines]{cancel}
%\usepackage{fancyvrb}
\usepackage{kotex}
\usepackage{paralist}
\usepackage{collectbox}
\usepackage{amsmath}
\usepackage{mathtools}
\usepackage{nccmath}
\usepackage{bm}

\xdefinecolor{dianablue}{rgb}{0.18,0.24,0.31}
\xdefinecolor{darkblue}{rgb}{0.1,0.1,0.7}
\xdefinecolor{darkgreen}{rgb}{0,0.5,0}
\xdefinecolor{darkgrey}{rgb}{0.35,0.35,0.35}
\xdefinecolor{darkorange}{rgb}{0.8,0.5,0}
\xdefinecolor{darkred}{rgb}{0.7,0,0}
\definecolor{darkgreen}{rgb}{0,0.6,0}
\definecolor{mauve}{rgb}{0.58,0,0.82}




\title[]{Lecture 4: Random Variable, Part II}
\author{Yi, Yung (이융)}
\institute{EE210: Probability and Introductory Random Processes\\ KAIST EE}
\date{MONTH DAY, 2021}

\input{../mymath}

\begin{document}

%itemshape
\setbeamertemplate{itemize item}{\scriptsize\raise1.25pt\hbox{\donotcoloroutermaths$\bullet$}}
\setbeamertemplate{itemize subitem}{\tiny\raise1.5pt\hbox{\donotcoloroutermaths$\circ$}}
\setbeamertemplate{itemize subsubitem}{\tiny\raise1.5pt\hbox{\donotcoloroutermaths$\blacktriangleright$}}
%default value for spacing
\plitemsep 0.1in
\pltopsep 0.03in
\setlength{\parskip}{0.15in}
%\setlength{\parindent}{-0.5in}
\setlength{\abovedisplayskip}{0.07in}
\setlength{\mathindent}{0cm}
\setbeamertemplate{frametitle continuation}{[\insertcontinuationcount]}

\setlength{\leftmargini}{0.5cm}
\setlength{\leftmarginii}{0.5cm}

\setlength{\fboxrule}{0.05pt}
\setlength{\fboxsep}{5pt}


\begin{frame}
  \titlepage
\end{frame}

% Uncomment these lines for an automatically generated outline.
\begin{frame}{Outline}
% \tableofcontents
\plitemsep 0.1in
\bci
\item Continuous Random Variable

\item PDF (Probability Density Function)

\item CDF (Cumulative Distribution Function)

\item Exponential and Normal Distribution

\item Joint PDF, Conditional PDF

\item Bayes' rule for continous and even mixed cases

\eci
\end{frame}

% START START START START START START START START START START START START START

%%%%%%%%%%%%%%%%%%%%%%%%%%%%%%%%%%%%%%%%%%%%%%%%%%%%%%
\begin{frame}{Roadmap}

\plitemsep 0.1in

\bci [$\circ$]
\item \bluef{Famous discrete random variables used in the community

- Bernoulli, Uniform, Binomial, Geometric, Poisson, etc. }

\item \bluef{Summarizing a random variable: Expectation and Variance}

\item \bluef{Functions of a single random variable}, \bluef{Functions of multiple random variables} 

\item \bluef{Conditioning for random variables}, \bluef{Independence for random variables} 

\item \redf{Continuous random variables

- Normal, Uniform, Exponential, etc. }

\item Bayes' rule for random variables
\eci 

\end{frame}

%%%%%%%%%%%%%%%%%%%%%%%%%%%%%%%%%%%%%%%%%%%%%%%%%%%%%%
\begin{frame}{Continuous RV and Probability Density Function}
\small

\smallskip

\onslide<2->{- Many cases when random variable have ``continuous values", e.g., velocity of a car}
\onslide<3->{
\myblock{Continuous Random Variable}
{
A rv $X$ is \redf{continuous} if $\exists$ a function $f_X,$ called \redblk{probability density function (PDF)}, s.t. 
\vspace{-0.3cm}
$$
\cprob{X \in B} = \int_B \fx dx
$$
\vspace{-0.5cm}
}}
\vspace{-0.5cm}
\onslide<4->{- All of the concepts and methods (expectation, PMFs, and conditioning) for discrete rvs have continuous counterparts}
\vspace{-0.2cm}
\mytwocols{0.4}
{
\onslide<5->{
\includegraphics[width=0.65\textwidth]{L4_pmf_ex.png}

\plitemsep 0.01in
\bci 
\item $\cprob{a \le X \le b} = \sum_{x: a\le x \le b} \px$

\item $\px \ge 0,$ $\sum_{x} \px = 1$
\eci} 
}
{
\onslide<6->{
\includegraphics[width=0.85\textwidth]{L4_pdf_ex.png}

\plitemsep 0.01in
\bci 
\item $\cprob{a \le X \le b} = \int_{a}^b \fx dx$

\item $\fx \ge 0,$ $\int_{-\infty}^{\infty} \fx dx= 1$
\eci} 
}
\end{frame}

%%%%%%%%%%%%%%%%%%%%%%%%%%%%%%%%%%%%%%%%%%%%%%%%%%%%%%
\begin{frame}{PDF and Examples}


\mytwocols{0.7}
{
\onslide<1->{
\includegraphics[width=0.65\textwidth]{L4_pdf_delta.png}

\bigskip

\plitemsep 0.1in
\bci 
\item $\cprob{a \le X \le a + \delta} \approx$ \redblank{2}{$f_X(a) \cdot \delta$}

\item \onslide<3->{$\cprob{X = a} = 0$}
\eci 
}
}
{
Examples

\onslide<4->{
\bigskip

\includegraphics[width=0.8\textwidth]{L4_pdf_uniform_ex.png}
}
}
\end{frame}

%%%%%%%%%%%%%%%%%%%%%%%%%%%%%%%%%%%%%%%%%%%%%%%%%%%%%%
\begin{frame}{Expectation and Variance}


\includegraphics[width=0.3\textwidth]{L4_uniform_ex.png}

\bigskip

\plitemsep 0.1in
\bci 
\item $\expect{X} = \int_{-\infty}^{\infty} x \fx dx$ = \onslide<2->{$\int_{a}^{b} \frac{x}{b-a} dx = \frac{1}{b-a}\frac{b^2 - a^2}{2} = \frac{b+a}{2}$}

\item $\expect{X^2} = \int_{-\infty}^{\infty} x^2 \fx dx$ = \onslide<3->{$\int_{a}^{b} \frac{x^2}{b-a} dx = \frac{1}{b-a}\frac{b^3 - a^3}{3} = \frac{a^2 + ab + b^2}{3}$}

\item $\var{X}$ = \onslide<4->{$\frac{a^2 + ab + b^2}{3} - \frac{a^2 + 2ab + b^2}{4}$}
\eci 


\end{frame}

%%%%%%%%%%%%%%%%%%%%%%%%%%%%%%%%%%%%%%%%%%%%%%%%%%%%%%
\begin{frame}{Cumulative Distribution Function (CDF)}

\mytwocols{0.8}
{
\small
\vspace{0.1in}
\plitemsep 0.1in
\bci 
\item<1-> Discrete: PMF, Continuous: PDF
\item<2-> Can we describe all rvs with a single mathematical concept? 
\onslide<3->{\aleq{
\Fx &= \cprob{X \le x} = \cr
& \begin{cases}
\sum_{k \le x} p_X(k), & \text{discrete}\cr
\int_{-\infty}^x f_X(t) dt, & \text{continuous}
\end{cases}
}}
\item<4-> always well defined, because we can always compute the probability for the event $\{X \le x \}$

\item<5-> CCDF (Complementary CDF): $\cprob{X > x }$
\eci 
}
{
\onslide<6->{\includegraphics[width=0.8\textwidth]{L4_cdf_ex1.png}}

\onslide<7->{\includegraphics[width=0.8\textwidth]{L4_cdf_ex2.png}}
}
\end{frame}

%%%%%%%%%%%%%%%%%%%%%%%%%%%%%%%%%%%%%%%%%%%%%%%%%%%%%%
\begin{frame}{CDF Properties}


\bigskip

\plitemsep 0.3in
\bci 
\item<2-> Non-decreasing

\item<3-> $F_X(x)$ tends to 1, as $x \rightarrow \infty$

\item<3-> $F_X(x)$ tends to 0, as $x \rightarrow -\infty$

\eci 
\bigskip

\onslide<4->{Now, let's look at famous continuous random variables popularly used in our life.} 
\end{frame}

%%%%%%%%%%%%%%%%%%%%%%%%%%%%%%%%%%%%%%%%%%%%%%%%%%%%%%
\begin{frame}{Exponential RV with parameter $\lambda >0$: $\exp(\lambda)$}

\plitemsep 0.02in
\bci 
\item<2-> A rv $X$ is called \redf{exponential with $\lambda$,} if
\aleq{
\fx = 
 \begin{cases}
 \lambda \elambdax, & x \ge 0 \cr
 0, & x <0
 \end{cases}
&\quad \text{or} \quad
\Fx = 1- \elambdax
}
\item<3-> Models a waiting time
\item<4-> CCDF $\cprob{X \ge x} = \elambdax$ (waiting time decays exponentially)
\item<5-> $\expect{X} = 1/\lambda$, $\expect{X^2} = 2/\lambda^2$, $\var{X} = 1/\lambda^2$
\item<6-> \redf{(Q)} What is the discrete rv which models a waiting time?  
\eci 

\vspace{-0.5cm}
\raggedleft
\onslide<2->{\includegraphics[width=0.5\textwidth]{L4_exp_pdf.png}}
%\mypic{0.9}{L4_exp_pdf.png}

\end{frame}

%%%%%%%%%%%%%%%%%%%%%%%%%%%%%%%%%%%%%%%%%%%%%%%%%%%%%%
\begin{frame}{Modeling Waiting Time? A Discrete Twin (1)}

\plitemsep 0.05in
\bci 
\item<2-> A discrete twin for modeling waiting times is \redf{geometric} rvs. 

\item<3-> Models a system evolution over time: Continuous time vs. Discrete time. In many cases, continuous case is the some type of \redblank{4}{limit} of its corresponding discrete case.  

\item<4-> Can you make mathematical description, where geometric and exponential rvs meet each other in the limit? 

\item<5-> \redf{Key idea.}
\bci
\item Continuous system: Discrete system with \redblank{6}{infinitely many slots whose duration is infinitely small.} 

\eci

\item<7-> limiting system: $X_{exp}(\lambda)$ with CDF $F_{exp}(\cdot)$

\item<8-> $n$-th system: $X^n_{geo}(p_n)$  with CDF $F^n_{geo}(\cdot)$

\eci 
% \raggedleft
% \includegraphics[width=0.5\textwidth]{L4_exp_pdf.png}
\end{frame}

%%%%%%%%%%%%%%%%%%%%%%%%%%%%%%%%%%%%%%%%%%%%%%%%%%%%%%
\begin{frame}{Modeling Waiting Time? A Discrete Twin (2)}

\myvartwocols{0.7}{0.65}{0.33}
{
For a given $x>0,$

\plitemsep 0.05in
\bci 
\item<2-> Define $\delta = \frac{x}{n}$ (a slot length in the $n$-th system)

\item<3-> Remember
\aleq{
F_{exp}(x) &= 1- \elambdax \cr
F^n_{geo}(n) &= 1- (1-p_n)^n
}
\item<4-> Choose $p_n = 1-e^{-\lambda \delta} = 1-e^{-\lambda \frac{x}{n}}.$

\item<5-> As $n \rightarrow \infty,$ the slot length $\delta \rightarrow 0$ thus $p_n \rightarrow 0$ 

\item<5-> The CDF values of exponential and $n$-th geometric rvs become equal whenever $x =\delta, 2\delta, 3\delta, \ldots,$ i.e., 
\aleq{
F_{exp}(n\delta) = F^n_{geo}(n), \quad n=1, 2, \ldots
}

\eci
}
{
% \raggedleft
\includegraphics[width=0.95\textwidth]{L4_exp_geo.png}
\small
\bci
\item<6-> As $n$ grows, the number of slots grows, but the success probability over one slot decreases, so that everything is balanced up. 
\item<7-> As $n$ grows, $F^n_{geo}(n)$ approaches $F_{exp}(n\delta).$
\eci
}
\end{frame}

%%%%%%%%%%%%%%%%%%%%%%%%%%%%%%%%%%%%%%%%%%%%%%%%%%%%%%
\begin{frame}{Normal (also called Gaussian) Random Variable}

Why important?

\plitemsep 0.4in
\bci 
\item Central limit theorem (중심극한정리)

- One of the most remarkable findings in the probability theory

\item Convenient analytical properties

\item Modeling aggregate noise with many small, independent noise terms

\eci
\end{frame}

%%%%%%%%%%%%%%%%%%%%%%%%%%%%%%%%%%%%%%%%%%%%%%%%%%%%%%
\begin{frame}{Normal: PDF, Expectation, Variance}

\mytwocols{0.4}
{
\plitemsep 0.1in
\bci 
\item<1-> Standard Normal $N(0,1)$
\aleq{
\fx & = \frac{1}{\sqrt{2\pi}} e^{-x^2/2}
}
\item<1-> $\expect{X} = 0$

\item<1-> $\var{X} = 1$
\eci
}
{
\plitemsep 0.1in
\bci 
\item<2-> General Normal $N(\mu, \sigma^2)$
\aleq{
\fx & = \frac{1}{\sigma\sqrt{2\pi}} e^{-(x-\mu)^2/2\sigma^2}
}

\item<2-> $\expect{X} = \mu$

\item<2-> $\var{X} = \sigma^2$

\eci
}

\medskip

\onslide<3->{
Need to check:

- a legitimate PDF or not

- expectation/variance
}
\end{frame}

%%%%%%%%%%%%%%%%%%%%%%%%%%%%%%%%%%%%%%%%%%%%%%%%%%%%%%
\begin{frame}{Normal: Useful Property}

\plitemsep 0.1in
\bci 
\item<2-> Linear transformation preserves normality

\myblock{Linear transformation of Normal}
{
If $X \sim  Norm(\mu, \sigma^2) $, then for $a \neq 0$ and $b$ $Y = aX +b \sim Norm(a\mu +b,a^2 \sigma^2).$ 
}

\item<3-> Thus, every normal rv can be \redblank{4}{standardized}: 

If $X \sim  Norm(\mu, \sigma^2)$, then 
\redblank{4}{$Y = \frac{X-\mu}{\sigma}$} $\sim Norm(0,1)$

\item<5-> Thus, we can make the \redf{table} which records the following CDF values:
\aleq{
\Phi(y) = \cprob{Y \le y} = \cprob{Y < y} = \frac{1}{\sqrt{2\pi}} \int_{-\infty}^y e^{-t^2/2} dt
}

\eci
\end{frame}

%%%%%%%%%%%%%%%%%%%%%%%%%%%%%%%%%%%%%%%%%%%%%%%%%%%%%%
\begin{frame}{Example}

\medskip

\mytwocols{0.7}
{
\plitemsep 0.1in
\bci 
\item<1-> Annual snowfall $X$ is modeled as $Norm(60,20^2).$ What is the probability that this year's snowfall is at least 80 inches?

\item<2-> $Y = \frac{X-60}{20}.$
\onslide<3->{
\aleq{
\cprob{X \ge 80} &= \cprob{Y \ge \frac{80-60}{20}} \cr
&= \cprob{Y \ge 1} = 1 - 
\Phi(1) \cr
&= 1- 0.8413 = 0.1587
}
}
\eci
}
{
\includegraphics[width=0.8\textwidth]{L4_normal_table.png}
}
\end{frame}

%%%%%%%%%%%%%%%%%%%%%%%%%%%%%%%%%%%%%%%%%%%%%%%%%%%%%%
\begin{frame}{Roadmap}

\plitemsep 0.05in

\bci [$\circ$]
\item \bluef{Famous discrete random variables used in the community

- Bernoulli, Uniform, Binomial, Geometric, Poisson, etc. }

\item \bluef{Summarizing a random variable: Expectation and Variance}

\item \magenf{Functions of a single random variable}, \magenf{Functions of multiple random variables} 

\item \magenf{Conditioning for random variables}, \magenf{Independence for random variables} 

\item \redf{Continuous random variables}

- Normal, Uniform, Exponential, etc. 

\item Bayes' rule for random variables

\eci 

\medskip

** Continuous counterparts are intuitively understandable. So, we will be quick at reviewing them.

\end{frame}

%%%%%%%%%%%%%%%%%%%%%%%%%%%%%%%%%%%%%%%%%%%%%%%%%%%%%%
\begin{frame}{Continuous: Joint PDF and CDF (1)}

\myblock{Jointly Continuous }
{
Two continuous rvs are \redblank{2}{jointly continuous} if a non-negative function $\fxy$ (called joint PDF) satisfies: for \redblk{every} subset $B$ of the two dimensional plane,
\aleq{
\cprob{(X,Y) \in B} = \iint_{(x,y) \in B} \fxy dx dy
}
\vspace{-0.3cm}
}

\plitemsep 0.1in
\bce 
\item<3-> The joint PDF is used to calculate probabilities  
$$\cprob{(X,Y) \in B} = \iint_{(x,y) \in B} \fxy dx dy$$

Our particular interest: $B = \{(x,y) \mid a \le x \le b, c \le y \le d \}$
\ece

\end{frame}

%%%%%%%%%%%%%%%%%%%%%%%%%%%%%%%%%%%%%%%%%%%%%%%%%%%%%%
\begin{frame}{Continuous: Joint PDF and CDF (2)}

\plitemsep 0.1in
\bce
\item<2->[2.] The marginal PDFs of $X$ and $Y$ are from the joint PDF as: 
$$
\fx = \int_{-\infty}^\infty \fxy dy, \quad \fy = \int_{-\infty}^\infty \fxy dx
$$

\item<3->[3.] The joint CDF is defined by $\Fxy = \cprob{X \le x, Y \le y},$ and determines the joint PDF as:
$$
\fxy = \frac{\partial^2 F_{x,y}}{\partial x \partial y} (x,y)
$$

\item<4->[4.] A function $g(X,Y)$ of $X$ and $Y$ defines a new random variable, and
$$
\expect{g(X,Y)} = \int_{-\infty}^\infty \int_{-\infty}^\infty g(x,y) \fxy dx dy
$$
\ece

\end{frame}

%%%%%%%%%%%%%%%%%%%%%%%%%%%%%%%%%%%%%%%%%%%%%%%%%%%%%%
\begin{frame}{Continuous:  Conditional PDF given an event}

\mytwocols{0.7}
{
* Conditional PDF, given an event 

\medskip

\plitemsep 0.15in
\bci 
\item<2-> $\fx \cdot \delta \approx \cprob{x \le X \le x+\delta}$
$\fxA \cdot \delta \approx \cprob{x \le X \le x+\delta | A}$

\item<3-> $\cprob{X \in B} = \int_B \fx dx$
$\cprob{X \in B | A} = \int_B \fxA dx$

\medskip
\redf{Note:} $A$ is an event, but $B$ is a subset that includes the possible values which can be taken by the rv $X.$

\item<4-> $\int \fxA = 1$
\eci
}
{
* Conditional PDF, given $X \in B$

\medskip

\onslide<5->{
\aleq{
\cprob{x \le X \le x+ \delta | X \in B} \approx f_{X|X \in B}(x) \cdot \delta
}
\aleq{
f_{X|X \in B}(x) = 
\begin{cases}
0, & \text{if} \quad x \notin B \cr
\frac{\fx}{\cprob{B}}, &\text{if} \quad x \in B
\end{cases}
}}

\onslide<6->{\redf{(Q)} In the discrete, we consider the event $\{X = x\}$, not $\{X \in B\}.$ Why?}

}
\end{frame}

%%%%%%%%%%%%%%%%%%%%%%%%%%%%%%%%%%%%%%%%%%%%%%%%%%%%%%
\begin{frame}{Continuous:  Conditional Expectation}

\mytwocols{0.8}
{
\plitemsep 0.15in

\onslide<1->{$A = \{\frac{a+b}{2} \le X \le b \}$}

%\smallskip

\centering
\onslide<1->{\mypic{0.65}{L4_cond_ex1.png}}
\only<1>{\mypic{0.65}{L4_cond_ex2.png}}
\only<2->{\mypic{0.65}{L4_cond_ex3.png}}

}
{
\bci 
\item<3-> $\expect{X} = \int x\fx dx$
$\expect{X | A} = \int x\fxA dx$

\item<4-> $\expect{g(X)} = \int g(x)\fx dx$
$\expect{g(X) | A} = \int g(x)\fxA dx$
\eci

\onslide<5->{
\aleq{
\expect{X|A} = \onslide<6->{\int_{(a+b)/2}^b x \frac{2}{b-a} dx = \frac{a}{4} + \frac{3b}{4}}
}
}
\onslide<5->{
\aleq{
\expect{X^2|A} = \onslide<6->{\int_{(a+b)/2}^b x^2 \frac{2}{b-a} dx =} 
}}
}
\end{frame}

%%%%%%%%%%%%%%%%%%%%%%%%%%%%%%%%%%%%%%%%%%%%%%%%%%%%%%
\begin{frame}{Exponential RV: Memoryless}

\plitemsep 0.1in
\bci 
\item<1-> Exponential rv is a continous counterpart of geometric rv.
\item<2-> Thus, expected to be memeoryless.

\onslide<3->{
\bigskip
\bluef{Definition.} A random variable $X$ is called \redblk{memoryless} if, for any $n,m \ge 0,$ $$\cprob{X > n+m | X > m} = \cprob{X > n}$$
}

\item<4-> \redf{Proof.} Note that $\cprob{X > x} = \elambdax.$ 
\onslide<5->{
Then, 
\aleq{
\cprob{X > n+m | X > m} = \frac{\cprob{X > n+m}}{\cprob{X > m}} = \frac{e^{-\lambda(n+m)}}{e^{-\lambda m}} = e^{-\lambda n} = \cprob{X > n}
}
}

\eci

\end{frame}

%%%%%%%%%%%%%%%%%%%%%%%%%%%%%%%%%%%%%%%%%%%%%%%%%%%%%%
\begin{frame}{Total Probability/Expectation Theorem}

Partition of $\Omega$ into $A_1,A_2,A_3, \ldots$

\medskip

\mytwocols{0.7}
{
\small
* Discrete case
\medskip

\begin{block}<2->{Total Probability Theorem}
\aleq{
 \px &= \sum_{i} \cprob{A_i} \cprob{X=x | A_i} \cr
 &= \sum_{i} \cprob{A_i} p_{X|A_i}(x) 
}
 \end{block}

\begin{block}<2->{Total Expectation Theorem}
\aleq{
 \expect{X} = \sum_{i} \cprob{A_i} \expect{X | A_i}
}
\end{block}
}
{
\small
* Continuous case
\medskip

\begin{block}<3->{Total Probability Theorem}
\aleq{
 \redf{\fx} = \sum_{i} \cprob{A_i} \redf{f_{X|A_i}(x)}  
}
 \end{block}

\begin{block}<4->{Total Expectation Theorem}
\aleq{
 \expect{X} = \sum_{i} \cprob{A_i} \expect{X | A_i}
}
\end{block}
}

\end{frame}

%%%%%%%%%%%%%%%%%%%%%%%%%%%%%%%%%%%%%%%%%%%%%%%%%%%%%%
\begin{frame}{Example }

\mytwocols{0.8}
{

\bigskip


\plitemsep 0.1in
\bci 
\item<1-> Your train's arrival every quarter hour (0, 15min, 30min, 45min). 

\item<1-> Your arrival $\sim$ uniform(7:10, 7:30) am. 

\item<2-> What is the PDF of waiting time for the first train?

\item<3-> $X:$ your arrival time, $Y:$ waiting time.

\item<4-> The value of $X$ makes a different waiting time. So, consider two events:

\medskip
$A = \{\text{7:10} \leq X \leq \text{7:15} \}$

\medskip
$B = \{\text{7:15} \leq X \leq \text{7:30} \}$

%\item Then, using the total probability theorem, 
\eci
}
{
\small
\mypic{0.85}{L4_total_prob_ex.png}
\aleq{
\onslide<5->{&\fy  = \cprob{A} \fyA + \cprob{B} f_{Y|B}(y)} \cr
&\onslide<6->{\fy = {1 \over 4} {1 \over 5} +{3 \over 4} {1 \over 15} = {1 \over 10},} \quad \onslide<5->{\text{for } 0 \le y \le 5} \cr 
&\onslide<7->{\fy = {1 \over 4} 0 +{3 \over 4} {1 \over 15} = {1 \over 20},} \quad \onslide<5->{\text{for } 5 < y \le 15}
}
}

\end{frame}

%%%%%%%%%%%%%%%%%%%%%%%%%%%%%%%%%%%%%%%%%%%%%%%%%%%%%%
\begin{frame}{Continuous:  Conditional PDF given a RV}

\mytwocols{0.8}
{
\medskip
\small
\plitemsep 0.1in
\bci 

\item $\pxcy = \frac{\pxy}{\py}$

\item<2-> Similarly, for $\fy >0,$
$$
\fxcy = \frac{\fxy}{\fy}
$$

\item<3-> Remember: For a fixed event $A,$ $\cprob{\cdot | A}$ is a legitimate probability law.

\item<4-> Similarly, For a fixed $y,$ $\fxcy$ is a legitimate PDF, since
$$
\int_{-\infty}^\infty \fxcy \redf{dx} = \frac{\int_{-\infty}^\infty \fxy dx}{\fy} = 1
$$
\eci
}
{
\medskip
\small
\plitemsep 0.01in
\bci 

\item<5-> \redf{Multiplication rule.}
\aleq{
\fxy &= \fy \cdot \fxcy \cr
&= \fx \fycx
}

\item<6-> \redf{Total prob./exp. theorem.}
\aleq{
\fx &= \int_{-\infty}^\infty \fy \fxcy dy \cr
\expect{X | Y=y} &= \int_{-\infty}^\infty x \fxcy dx \cr
\expect{X} &= \int_{-\infty}^\infty \fy \expect{X | Y=y} dy
}

\item<7-> \redf{Independence.}
$$
\fxy = \fx \fy, \quad \text{for all $x$ and $y$}
$$
\eci
}
\end{frame}

%%%%%%%%%%%%%%%%%%%%%%%%%%%%%%%%%%%%%%%%%%%%%%%%%%%%%%
\begin{frame}{Example: Stick-breaking (Ch 3. Prob 21)}

\mytwocols{0.7}
{
%\medskip
%\small
\plitemsep 0.1in
\bci 

\item<1-> Break a stick of length $l$ twice

- first break at $X \sim uniform[0.l]$

- second break at $Y \sim uniform[0,X]$

\item<2-> \redf{(Q)} What is $\expect{Y}$? 

\item<3-> Since $Y$ depends on $X,$ the total expectation theorem seems useful.
$$
\expect{Y} = \int_{-\infty}^\infty \fx \expect{Y| X=x} dx
$$

\eci
}
{
%\medskip
%\small
\plitemsep 0.1in
\bci 

\item<4-> Using the TET, 
\aleq{
\expect{Y} &= \int_{0}^l \frac{1}{l} \expect{Y| X=x} dx\cr
&= \int_{0}^l \frac{1}{l} \frac{x}{2} dx = \frac{l}{4}
}

\item<5-> $\fx$ and $\fycx$ seems easy to compute. Thus, 
\aleq{
\fxy &= \fx \fycx = {1 \over l} \cdot {1 \over x}
}
You can do many other things with the joint PDF. 
\eci
}


\end{frame}

%%%%%%%%%%%%%%%%%%%%%%%%%%%%%%%%%%%%%%%%%%%%%%%%%%%%%%
\begin{frame}{Roadmap}

\plitemsep 0.05in

\bci [$\circ$]
\item \bluef{Famous discrete random variables used in the community

- Bernoulli, Uniform, Binomial, Geometric, Poisson, etc. }

\item \bluef{Summarizing a random variable: Expectation and Variance}

\item \bluef{Functions of a single random variable}, \bluef{Functions of multiple random variables} 

\item \bluef{Conditioning for random variables}, \bluef{Independence for random variables} 

\item \bluef{Continuous random variables

- Normal, Uniform, Exponential, etc. }

\item \redf{Bayes' rule for random variables}
\eci 
\end{frame}

%%%%%%%%%%%%%%%%%%%%%%%%%%%%%%%%%%%%%%%%%%%%%%%%%%%%%%
\begin{frame}{Bayes Rule for Continuous}

\plitemsep 0.1in
\bci 

\item<1-> $X$: state/cause/original value $\rightarrow$ $Y$: result/resulting action/noisy measurement

\item<1-> Model: $\cprob{X}$ (prior) and $\cprob{Y | X}$ (cause $\rightarrow$ result)

\item<1-> Inference: $\cprob{X | Y}$?

 \bigskip
 \mytwocols{0.4}
 {
\onslide<2->{
\aleq{
\pxy & = \px \pycx \cr
     & = \py \pxcy \cr
\redf{\pxcy} & = \frac{\px \pycx}{\py}\cr
\py &= \sum_{x'} p_X(x')p_{Y|X}(y|x')
}
 }}
{

\aleq{
\onslide<2->{
\fxy & = \fx \fycx \cr
     & = \fy \fxcy \cr
\redf{\fxcy} & = \frac{\fx \fycx}{\fy}\cr
\fy &= \int f_X(x')f_{Y|X}(y|x') dx'
}
}
}

\eci 
\end{frame}

%%%%%%%%%%%%%%%%%%%%%%%%%%%%%%%%%%%%%%%%%%%%%%%%%%%%%%
\begin{frame}{Bayes Rule for Mixed Case}

\onslide<1->{$K$: discrete, $Y$: continuous}

\bigskip

\plitemsep 0.1in
\mytwocols{0.5}
{
\bci 

\item<2-> Inference of $K$ given $Y$
\aleq{
\onslide<3->{
\redf{p_{K|Y}(k|y)} &= \frac{\redf{p_{K}(k)} \bluef{f_{Y|K}(y|k)}}{\bluef{\fy}}\cr
\bluef{\fy} &= \sum_{k'} \redf{p_{K}(k')} \bluef{f_{Y|K}(y|k')}
}}
\eci
}
{
\bci 

\item<2-> Inference of $Y$ given $K$
\aleq{
\onslide<4->{
\bluef{f_{Y|K}(y|k)} &= \frac{\bluef{\fy} \redf{p_{K|Y}(k|y)}}{\redf{p_{K}(k)}} \cr
\redf{p_{K}(k)} &= \int \bluef{f_{Y}(y')} \redf{p_{K|Y}(k|y')} dy'
}}
\eci 
}

\end{frame}

%%%%%%%%%%%%%%%%%%%%%%%%%%%%%%%%%%%%%%%%%%%%%%%%%%%%%%
\begin{frame}{Example: Signal Detection (1)}


Inference of discrete $K$ given continuous $Y$: 
\aleq{
\redf{p_{K|Y}(k|y)} = \frac{\redf{p_{K}(k)} \bluef{f_{Y|K}(y|k)}}{\bluef{\fy}}, \quad \bluef{\fy} &= \sum_{k'} \redf{p_{K}(k')} \bluef{f_{Y|K}(y|k')}
}

\plitemsep 0.01in
\bci

\item<2-> $K$: -1, +1, original signal, equally likely. $p_{K}(1) = 1/2, p_{K}(-1) = 1/2.$
\item<3-> $Y$: measured signal with Gaussian noise, $Y= K+W,$ $W \sim N(0,1)$

\medskip
\item<4-> Your received signal = 0.7. What's your guess about the original signal? \onslide<6->{\redf{+1}}
\item<5-> Your received signal = -0.2. What's your guess about the original signal? \onslide<6->{\redf{-1}}

\eci 
\end{frame}

%%%%%%%%%%%%%%%%%%%%%%%%%%%%%%%%%%%%%%%%%%%%%%%%%%%%%%
\begin{frame}{Example: Signal Detection (2)}

\plitemsep 0.01in
\bci 

\item<1-> $Y|K=1 \sim N(1,1)$ and $Y|K=-1 \sim N(-1,1).$ 
\aleq{
\onslide<2->{f_{Y|K}(y|k) &= \frac{1}{\sqrt{2 \pi}} e^{-\frac{1}{2}(y-k)^2}, \quad k=1, -1} \cr
\onslide<3->{\fy &= \frac{1}{2} \frac{1}{\sqrt{2 \pi}} e^{-\frac{1}{2}(y+1)^2} + 
\frac{1}{2} \frac{1}{\sqrt{2 \pi}} e^{-\frac{1}{2}(y-1)^2}}
}

\item<4-> Probability that $K=1$, given $Y=y$? After some algebra, 

\mytwocols{0.3}
{
$$
p_{K|Y}(1|y) = \frac{1}{1+ e^{-2y}}
$$
}
{
\centering
\onslide<5->{\mypic{0.8}{L4_sig_detection.png}}
}

\eci 

\end{frame}

\begin{frame}{}
\vspace{2cm}
\LARGE Questions?

\end{frame}

\begin{frame}{Review Questions}

\bce[1)]
\item What is PDF  and CDF? 

\item Why do we need CDF? 

\item What are joint/marginal/conditional PDFs?

\item Explain memorylessness of exponential random variables.

\item Explain the version of Bayes' rule for continuous and mixed random variables.

\ece

\end{frame}

\end{document}
