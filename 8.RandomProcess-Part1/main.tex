
\def\slidemode{%
  \documentclass[fleqn,aspectratio=169]{beamer}
}
\def\handoutmode{%
  \documentclass[handout,fleqn,aspectratio=169]{beamer}
}


\def\slidemode{%
  \documentclass[fleqn,aspectratio=169]{beamer}
\usepackage{pgfpages}
}
\def\handoutmode{%
  \documentclass[handout,fleqn,aspectratio=169]{beamer}
\usepackage{pgfpages}
\pgfpagesuselayout{resize to}[a4paper,landscape,border shrink=5mm]
}


%\def\pdfmode{handoutmode}
\def\pdfmode{slidemode}

\csname\pdfmode\endcsname

\usepackage{pgfpages}
\pgfpagesuselayout{resize to}[a4paper,landscape,border shrink=5mm]

\mode<presentation>
{
  \usetheme{default}
  \usecolortheme{default}
  \usefonttheme{default}
  \setbeamertemplate{navigation symbols}{}
  \setbeamertemplate{caption}[numbered]
  \setbeamertemplate{footline}[frame number]  % or "page number"
  \setbeamercolor{frametitle}{fg=white}
  \setbeamercolor{footline}{fg=black}
} 

\usepackage[english]{babel}
%\usepackage[utf8x]{inputenc}
\usepackage{tikz}
\usepackage{courier}
\usepackage{array}
\usepackage{bold-extra}
% \usepackage{minted}
% \usepackage[thicklines]{cancel}
\usepackage{fancyvrb}
\usepackage{kotex}
\usepackage{paralist}
\usepackage{collectbox}
\usepackage{booktabs}
\usepackage{nccmath}
%\usepackage[fleqn]{amsmath}



\setbeamercolor{block body alerted}{bg=alerted text.fg!10}
\setbeamercolor{block title alerted}{bg=alerted text.fg!20}
\setbeamercolor{block body}{bg=structure!10}
\setbeamercolor{block title}{bg=structure!20}
\setbeamercolor{block body example}{bg=green!10}
\setbeamercolor{block title example}{bg=green!20}
\setbeamertemplate{blocks}[rounded][shadow]

\xdefinecolor{dianablue}{rgb}{0.18,0.24,0.31}
\xdefinecolor{darkblue}{rgb}{0.1,0.1,0.7}
\xdefinecolor{darkgreen}{rgb}{0,0.5,0}
\xdefinecolor{darkgrey}{rgb}{0.35,0.35,0.35}
\xdefinecolor{darkorange}{rgb}{0.8,0.5,0}
\xdefinecolor{darkred}{rgb}{0.7,0,0}
\definecolor{darkgreen}{rgb}{0,0.6,0}
\definecolor{mauve}{rgb}{0.58,0,0.82}


\newcommand{\progressbar}{
\pgfmathsetmacro{\theta}{360/\inserttotalframenumber*\inserttotalframenumber}
\begin{tikzpicture}[scale=0.035]
\fill[yellow] (0,0) circle (9);
\fill[red] (0,0) -- (9,0) arc (0:-\theta:9);
\fill[white] (0,0) circle (5);
\end{tikzpicture}
}

%\setbeamertemplate{footline}{\hfill \progressbar}

\title[]{Lecture 8: Random Processes, Part I}
\author{Yi, Yung (이융)}
\institute{EE210: Probability and Introductory Random Processes\\ KAIST EE}
\date{MONTH DAY, 2021}

\usetikzlibrary{shapes.callouts}

\input{../mymath}


\begin{document}

%itemshape
\setbeamertemplate{itemize item}{\scriptsize\raise1.25pt\hbox{\donotcoloroutermaths$\bullet$}}
\setbeamertemplate{itemize subitem}{\tiny\raise1.5pt\hbox{\donotcoloroutermaths$\circ$}}
\setbeamertemplate{itemize subsubitem}{\tiny\raise1.5pt\hbox{\donotcoloroutermaths$\blacktriangleright$}}
%default value for spacing
\plitemsep 0.1in
\pltopsep 0.03in
\setlength{\parskip}{0.15in}
%\setlength{\parindent}{-0.5in}
\setlength{\abovedisplayskip}{0.07in}
\setlength{\mathindent}{0cm}
\setbeamertemplate{frametitle continuation}{[\insertcontinuationcount]}

\setlength{\leftmargini}{0.5cm}
\setlength{\leftmarginii}{0.5cm}

\setlength{\fboxrule}{0.05pt}
\setlength{\fboxsep}{5pt}

\logo{\pgfputat{\pgfxy(0.11, 7.4)}{\pgfbox[right,base]{\tikz{\filldraw[fill=dianablue, draw=none] (0 cm, 0 cm) rectangle (50 cm, 1 cm);}\mbox{\hspace{-8 cm}\includegraphics[height=0.7 cm]{../kaist_ee.png}
}}}}

\begin{frame}
  \titlepage
\end{frame}

\logo{\pgfputat{\pgfxy(0.11, 7.4)}{\pgfbox[right,base]{\tikz{\filldraw[fill=dianablue, draw=none] (0 cm, 0 cm) rectangle (50 cm, 1 cm);}\mbox{\hspace{-8 cm}\includegraphics[height=0.7 cm]{../kaist_ee.png}
}}}}

% % Uncomment these lines for an automatically generated outline.
% \begin{frame}{Outline}
% % \tableofcontents
% \plitemsep 0.1in
% \bci
% \item 

% \item 
% \eci
% \end{frame}

% START START START START START START START START START START START START START

%%%%%%%%%%%%%%%%%%%%%%%%%%%%%%%%%%%%%%%%%%%%%%%%%%%%%%
\begin{frame}{Roadmap}

\plitemsep 0.1in

\bci 
\item A lot of applications in engineering systems


\bigskip

\item Basics on Random Process

\medskip
\item Bernoulli Process
\item Poisson Process
\item Use of Bernoulli and Poisson Processes

- Coding of both processes

- Merge and Split

\medskip
\item Markov Chain

\eci 

\end{frame}

%%%%%%%%%%%%%%%%%%%%%%%%%%%%%%%%%%%%%%%%%%%%%%%%%%%%%%
\begin{frame}{Roadmap}

\plitemsep 0.1in

\bci 
\item A lot of applications in engineering systems


\bigskip

\item \redf{Basics on Random Process}

\medskip
\item Bernoulli Process
\item Poisson Process
\item Use of Bernoulli and Poisson Processes

- Coding of both processes

- Merge and Split

\medskip
\item Markov Chain

\eci 

\end{frame}

%%%%%%%%%%%%%%%%%%%%%%%%%%%%%%%%%%%%%%%%%%%%%%%%%%%%%%
\begin{frame}{Things that evolve in time}


\mytwocols{0.75}
{
\plitemsep 0.1in
\bci 
\item<1-> Many probabilistic experiments that \redf{evolve in time}
\bci
\item<2-> Sequence of daily prices of a stock
\item<2-> Sequence of scores in football
\item<2-> Sequence of failure times of a machine
\item<2-> Sequence of traffic loads in Internet
\eci

\item Random process is a mathematical model for it.

%\item Each numerical value in the sequence is a random variable. 


\eci 
}
{
\centering
\mypic{0.8}{L8_bitcoin.png}
\mypic{0.8}{L8_nettraffic.png}
}
\end{frame}


%%%%%%%%%%%%%%%%%%%%%%%%%%%%%%%%%%%%%%%%%%%%%%%%%%%%%%
\begin{frame}{Random Process: A Sneak Peek}

\plitemsep 0.1in
\bci 

\item<2-> A random process is a \redf{sequence} of random variables indexed by \redf{time.}

\item<3-> Time: discrete or continuous

\item<4-> Notation

- $(X_t)_{t \in \set{T}}$ or $\Bl(X(t)\Bl)_{t \in \set{T}},$ where $\set{T} = \real$ (continuous) or $\set{T} = \{ 0, 1, 2, \ldots \}$ (discrete)

\onslide<5->{- For the discrete case, we also often use $(X_n)_{n \in \integer_+}.$} 

\onslide<6->{- We will use all of them, unless confusion arises. }

\item<7-> For a fixed time $t$, $X_t$ is a random variable.

\item<8-> The values that $X_t$ can take: discrete or continuous

\eci

\end{frame}

%%%%%%%%%%%%%%%%%%%%%%%%%%%%%%%%%%%%%%%%%%%%%%%%%%%%%%
\begin{frame}{4 Types of Random Processes}

- Types of time and value

\bigskip
\mytwocols{0.5}
{
\bigskip
\small
\plitemsep 0.1in
\bce[(a)] 
\item continuous time, continuous value
\item continuous time, discrete value
\item discrete time, continuous value
\item discrete time, discrete value
\ece

}
{
\centering
\mypic{0.95}{L8_RP_types.jpg}

}

\end{frame}

%%%%%%%%%%%%%%%%%%%%%%%%%%%%%%%%%%%%%%%%%%%%%%%%%%%%%%
\begin{frame}{Roadmap}

\plitemsep 0.1in

\bci 
\item A lot of applications in engineering systems


\bigskip

\item \bluef{Basics on Random Process}

\medskip
\item \redf{Bernoulli Process}
\item Poisson Process
\item Use of Bernoulli and Poisson Processes

- Coding of both processes

- Merge and Split

\medskip
\item Markov Chain

\eci 

\end{frame}


%%%%%%%%%%%%%%%%%%%%%%%%%%%%%%%%%%%%%%%%%%%%%%%%%%%%%%
\begin{frame}{Bernoulli Process}

\plitemsep 0.1in
\bci

\item<1-> At each minute, we toss a coin with probability of head $0<p<1.$

\bci
\item<2-> Sequence of lottery wins/looses
\item<2-> Customers (each second) to a bank
\item<2-> Clicks (at each time slot) to server
\eci


\item<3-> A sequence of \redf{independent Bernoulli trials} $X_1, X_2, \ldots, $

- We call index 1, 2, \ldots \redf{time slots} (or simply slots)

\mypic{0.5}{L8_bernoulli_ex.png}

\item<4-> Discrete time, discrete value

\item<5-> One of the simplest random processes

\item<5-> A type of ``arrival" process

\eci

\end{frame}

%%%%%%%%%%%%%%%%%%%%%%%%%%%%%%%%%%%%%%%%%%%%%%%%%%%%%%
\begin{frame}{Bernoulli Process as a Random Process}

\plitemsep 0.1in
\bci

\item<1-> \redf{Question.} We've already studied a sequence of Bernoulli rvs $X_1, X_2, \ldots, X_n.$ What's the difference?

\item<2-> \bluef{Physical difference:} \redf{infinite} sequence of $X_1, X_2, \ldots, .$
\bci
\item<3-> Sample space? set of all outcomes?
\item<4-> an outcome: an infinite sequence of sample values $x_1, x_2, \ldots,$ e.g., (0,1,1,0,0,1, \ldots)
\eci

\item<5-> \bluef{Semantic difference:} Understand $i$ in $X_i$ as time. Also, interesting questions from the random process point of view. 
\bci
\item<6-> Dependence: How $X_1, X_2, \ldots$ are related to each other as a time series
\item<7-> Long-term behavior: What is the fraction of times that a machine is idle?
\item<8-> Other interesting questions, depending on the target random process
\eci

\item<9-> Next: Key questions and answers about Bernoulli process
\eci

\end{frame}

%%%%%%%%%%%%%%%%%%%%%%%%%%%%%%%%%%%%%%%%%%%%%%%%%%%%%%
\begin{frame}{Number of arrivals and Time until the first arrival}

\mytwocols{0.6}
{
\plitemsep 0.1in

\onslide<1->{\redf{(Q1)} \# of arrivals in $n$ slots?}
\bci
\item<2-> $S_n = X_1 + X_2 +\cdots + X_n$

\item<3-> $S_n \sim Bin(n,p)$

\item<4-> $\cexpect{S_n} = np$, $\cvar{S_n} = np(1-p)$
\eci

\bigskip
\onslide<5->{\redf{(Q2)} \# of slots $T_1$ until the first arrival?}
\bci
\item<6-> $T_1 \sim Geom(p)$
\item<7-> $\cexpect{T_1} = 1/p,$ $\cvar{T_1} = \frac{1-p}{p^2}$
\eci
}
{
\plitemsep 0.1in
\bci
\item<8-> $T_1$ is geometric? \bluef{Memoryless}

\item<9-> Conditioned no first arrival at some time, what's the distribution of the remaining time until the first arrival? 

\item<9-> Still, geometric.

\item<10-> But, more than that, as we will see. Independence across time slots leads to many useful properties, allowing the quick solution of many problems. 

\eci

}

\end{frame}

%%%%%%%%%%%%%%%%%%%%%%%%%%%%%%%%%%%%%%%%%%%%%%%%%%%%%%
\begin{frame}{Memoryless and Fresh-start after Deterministic $n$}

\plitemsep 0.1in

\onslide<1->{\redf{(Q3)} $U=X_1 + X_2 \indep V=X_5+X_6$?}
\bci
\item<2-> Yes

\item<2-> Because $X_i$s are independent

\eci

\bigskip
\onslide<3->{\redf{(Q4)} The process $(X_n)_{n=6}^\infty$?}
\bci
\item<4-> $(X_1, \ldots, X_5) \indep$  $(X_n)_{n=6}^\infty$
\item<5-> \redf{Fresh-start} after a deterministic time $n$
\item<6-> If you watch the on-going Bernoulli process$(p)$ from some time $n,$ you still see the same Bernoulli process$(p).$
\eci

\end{frame}
%%%%%%%%%%%%%%%%%%%%%%%%%%%%%%%%%%%%%%%%%%%%%%%%%%%%%%
\begin{frame}{Fresh-start after Random $N$ (1)}

\onslide<1->{\redf{(Q5)} The process $(X_{N},X_{N+1}, X_{N+2}, \ldots)$? Fresh-start even after \bluef{random} $N$?}
 
\plitemsep 0.1in
\bci
\item<2-> Examples of $N$
\mytwocols{0.27}
{
\small
\bce[\bf E1.]
\item<3-> Time of 3rd arrival
\item<4-> First time when 3 consecutive arrivals have been observed
\item<5-> Time just before 3 consecutive arrivals
\ece
}
{
\vspace{-1cm}
\centering
\mypic{0.8}{L8_randomtime.png}
}

\item<6-> Difference of $N$ from $n$

- The time when I watch the on-going Bernoulli process is \redf{random.} 

\eci

\end{frame}

%%%%%%%%%%%%%%%%%%%%%%%%%%%%%%%%%%%%%%%%%%%%%%%%%%%%%%
\begin{frame}{Fresh-start after Random $N$ (2)}


\redf{(Q5)} The process $(X_{N},X_{N+1}, X_{N+2}, \ldots)$? Fresh-start even after random $N$?
 
\plitemsep 0.05in
\bci
\item Examples of $N$
\mytwocols{0.27}
{
\small
\bce[\bf E1.]
\item Time of 3rd arrival
\item First time when 3 consecutive arrivals have been observed
\item Time just before 3 consecutive arrivals
\ece
}
{
\vspace{-1cm}
\centering
\mypic{0.7}{L8_randomtime.png}
}

\item<2->[\bf E1.] When I watch the process, $N$ has been already determined. \redf{Yes}
\item<3->[\bf E2.] Same as {\bf E1.} \redf{Yes}
\item<4->[\bf E3.] Need the future knowledge. `111' does not become random.  \redf{No}

\medskip
\item<5-> The question of $N=n$? can be answered just from the knowledge about $X_1, X_2, \ldots, X_n$? Then, Yes! (see pp. 301 for more formal description)

\eci
\end{frame}

%%%%%%%%%%%%%%%%%%%%%%%%%%%%%%%%%%%%%%%%%%%%%%%%%%%%%%
\begin{frame}{Distribution of Busy Periods}

\plitemsep 0.05in
\bci
\item Regard an arrival as business of a server

\item<2-> First busy period $B_1$: starts with the first busy slot and ends just before the first subsequent idle slot

\onslide<2->{
\begin{center}
\mypic{0.4}{L8_busy_idle.png}
\end{center}
}

\item<3-> \redf{(Q6)} Distribution of $B_1$?


\item<4-> $N$: time of the first busy slot. Fresh-start after $N.$

\item<5-> $B_1$ is geometric with parameter $(1-p)$

\item<6-> Question: What about the second busy period $B_2$? $B_3, B_4$?

\eci
\end{frame}

%%%%%%%%%%%%%%%%%%%%%%%%%%%%%%%%%%%%%%%%%%%%%%%%%%%%%%
\begin{frame}{Time of $k$-th arrival}

\plitemsep 0.1in
\bci
\item Time of the first arrival $Y_1 \sim geom(p)$

\item<2->[] \redf{(Q7)} Time of the $k$-th arrival $Y_k$?

\medskip
\myvartwocols{0.25}{0.6}{0.37}
{
\bigskip
\small
\onslide<3->{- $T_k = Y_k - Y_{k-1}$: $k$-th inter-arrival ($k\ge 2,$ $T_1 = Y_1$)}

\onslide<4->{- $Y_k = T_1 + T_2 + \ldots + T_k.$}
}
{
\centering
\onslide<3->{\mypic{0.8}{L8_interarrival.png}}
}

\item<5-> After each $T_k,$ the fresh-start occurs. 

\item<6-> $\{T_i\}$ are i.i.d. and $\sim geom(p)$

\item<7-> $\expect{Y_k} = \frac{k}{p},$ $\var{Y_k} = \frac{k(1-p)}{p^2}$
\eci
\end{frame}

%%%%%%%%%%%%%%%%%%%%%%%%%%%%%%%%%%%%%%%%%%%%%%%%%%%%%%
\begin{frame}{PMF of $Y_k$}

\plitemsep 0.1in
\bci
\item<1-> $Y_k = T_1 + T_2 + \ldots + T_k.$

\item<1-> $\{T_i\}$ are i.i.d. and $\sim geom(p)$
\aleq{
\onslide<2->{\cprob{Y_k = t} &= \cbprob{X_k = 1 \text{ and $k-1$ arrivals during the first $t-1$ slots} }}\cr
\onslide<3->{&=\cbprob{X_k = 1} \cdot \cbprob{\text{$k-1$ arrivals during the first $t-1$ slots}}}\cr
\onslide<4->{&= p \times {t-c \choose k-1} p^{k-1} (1-p)^{t-k} = {t-c \choose k-1} p^{k} (1-p)^{t-k}, \quad {t = k, k+1, \ldots}}
}

\item<5-> $Y_k$ is called \redf{Pascal rv} with parameter $(k,p).$

\item<6-> $Pascal(1,p)$ = $Geometric(p)$
\eci
\end{frame}

%%%%%%%%%%%%%%%%%%%%%%%%%%%%%%%%%%%%%%%%%%%%%%%%%%%%%%
\begin{frame}{Roadmap}

\plitemsep 0.1in

\bci 
\item A lot of applications in engineering systems


\bigskip

\item \bluef{Basics on Random Process}

\medskip
\item \bluef{Bernoulli Process}
\item \redf{Poisson Process}
\item Use of Bernoulli and Poisson Processes

- Coding of both processes

- Merge and Split

\medskip
\item Markov Chain

\eci 

\end{frame}


%%%%%%%%%%%%%%%%%%%%%%%%%%%%%%%%%%%%%%%%%%%%%%%%%%%%%%
\begin{frame}{Design of Continuous Analog of Bernoulli Process}

\plitemsep 0.15in
\bci
\item<1-> Very useful to both continuous and discrete random processes that are ``twins" and share the key properties.

\bci
\item<2-> Independence between what happens in a different time region

\item<3-> Memoryless and fresh-start property
\eci

\item<4-> Choose one of discrete or continuous versions for our modeling convenience.

\item<5-> \redf{Question.} How do we design the continuous analog of Bernoulli process?

\onslide<6->{- Key idea: Making it as a \redblank{7}{limiting system} of a sequence of Bernoulli processes}

\item<8-> Need a ``modeling sense" to make this possible. It's a good practice for engineers!
\eci
\end{frame}

%%%%%%%%%%%%%%%%%%%%%%%%%%%%%%%%%%%%%%%%%%%%%%%%%%%%%%
\begin{frame}{Key Design Idea to Develop a Continuous Twin (1)}

\plitemsep 0.1in
\bci
\item<1-> Continuous twin

\bci
\item<2-> Key point: Understand the number of arrivals over a given interval $[0,\tau].$

\item<3-> Assume that it has some arrival rate $\lambda$ (\# of arrivals/unit time).

\item<4-> We know how to handle Bernoulli process with discrete time slots. 
\eci


\item<5-> Divide $[0,\tau]$ into slots whose length $=\delta.$ Then, $n= \# \text{ of slots} = \frac{\tau}{\delta}.$ 
\begin{center}
\mypic{0.4}{L8_tau_delta.jpg}
\end{center}

\item<6-> What's the limit as $\delta \rightarrow 0$ (equivalently, $n \rightarrow \infty$)

\eci
\end{frame}

%%%%%%%%%%%%%%%%%%%%%%%%%%%%%%%%%%%%%%%%%%%%%%%%%%%%%%
\begin{frame}{Key Design Idea to Develop a Continuous Twin (2)}

\begin{center}
\mypic{0.4}{L8_tau_delta.jpg}
\end{center}
\vspace{-0.8cm}
\plitemsep 0.1in
\bci
\item<1-> Now, our design idea: during one time slot of length $\delta$, 
\mytwocols{0.3}
{
\aleq{
\onslide<2->{\cprob{\text{1 arrival}} &\propto \text{arrival rate and slot length}} \cr
\onslide<3->{\cprob{\text{$\ge$2 arrivals}} & \propto \text{something, but very small }}\cr
& \onslide<4->{\quad \text{   for small sloth length}\cr
\cprob{\text{0 arrival}} &= 1- \cprob{\text{1 arrival or $\ge$2 arrivals} }} \cr
}
}
{
\aleq{
\onslide<5->{\cprob{\text{1 arrival}} &= \lambda \delta}  \onslide<8->{+o(\delta)}\cr
\onslide<6->{\cprob{\text{$\ge$2 arrivals}} &=} \onslide<6-7>{0} \onslide<8->{o(\delta)}\cr
\onslide<7->{\cprob{\text{0 arrival}} &= 1-  \lambda \delta} \onslide<8->{+ o(\delta)} 
}
}
\item<9-> $o(\delta)$: some function that goes to zero faster than $\delta$ goes to zero. 

- Thus, for very small $\delta,$ $o(\delta)$ becomes negligible. 

- Example: $o(\delta)= \delta^\alpha,$ where any $\alpha >1$



\eci
\end{frame}

%%%%%%%%%%%%%%%%%%%%%%%%%%%%%%%%%%%%%%%%%%%%%%%%%%%%%%
\begin{frame}{Key Design Idea to Develop a Continuous Twin (3)}

\begin{center}
\mypic{0.4}{L8_tau_delta.jpg}
\end{center}
\vspace{-0.8cm}
\plitemsep 0.1in
\bci
\item<1-> Our interest: Prob. of $k$ arrivals over $[0,\tau]$

\item<2-> Given ``small" $\delta,$ \# of arrivals $\sim Binomial(n,p),$ where $n = \tau/\delta$ and $p =\lambda \delta$

\item<3-> As $\delta \rightarrow \infty$, $np = \tau/\delta \times \lambda \delta = \lambda \tau.$

\item<4-> \# of arrivals over $[0,\tau]$, $\sim Poisson(\lambda \tau)$

\item<5-> This is a continuous twin process of Bernous process, which we call \redf{Poisson process.}

\eci
\end{frame}


%%%%%%%%%%%%%%%%%%%%%%%%%%%%%%%%%%%%%%%%%%%%%%%%%%%%%%
\begin{frame}{Poisson Process: Formalism}

\plitemsep 0.1in
\bci
\item<2-> $N_s$: number of arrivals over the interval $[0,s].$

\item<3-> \redf{(Independence)} If $s < t,$ the number $N_t - N_s$ of arrivals over $[s,t]$ is independent of the times of arrivals during $[0,s].$

\onslide<4->{- Thus, $N_s$ can be a random variable over \bluef{any} interval of length $s.$}

\item<5-> \redf{(Small interval probability)} The probabilities $\cprob{k,s}$ satisfy:
\aleq{
\cprob{0,s} &= 1-  \lambda \tau + o(s) \cr
\cprob{1, s} &= \lambda s + o_1(s)\cr
\cprob{k,s} &=  o_k(s)\quad \text{for $k=2, 3, \ldots$},
}
where 
\aleq{
\lim_{s \rightarrow 0} \frac{o(s)}{s} = 0, \quad \lim_{s \rightarrow 0} \frac{o_k(s)}{s} = 0
}

\eci
\end{frame}



%%%%%%%%%%%%%%%%%%%%%%%%%%%%%%%%%%%%%%%%%%%%%%%%%%%%%%
\begin{frame}{Poisson Process: $\cprob{k,\tau},$ $N_\tau,$ and $T$}

\plitemsep 0.1in
\bci
\item<2-> \redf{(Q1)} Number of arrivals of any interval with length $\tau$ $\sim Poisson(\lambda \tau),$ i.e., 
\aleq{
\cprob{k,\tau} = e^{-\lambda \tau}\frac{(\lambda \tau)^k}{k!}, \quad k=0,1,2, \ldots
}

\item<3-> $\expect{N_\tau} = \lambda \tau$ and $\var{N_\tau} = \lambda \tau$ 

\item<4-> \redf{(Q2)} Time of first arrival $T$
\aleq{
\onslide<5->{F_T(t) &= \cprob{T \le t} = 1- \cprob{T >t} = 1- P(0,t) = 1- e^{-\lambda t}}\cr
\onslide<6->{f_T(t) &= \frac{dF_T(t)}{dt} = \lambda e^{-\lambda t}, \quad t \ge 0.}
}

\item<7-> $T \sim expo(\lambda).$ Thus $\expect{T} = 1/\lambda$ and $\var{T} = 1/\lambda^2$ 

- Continuous twin of geometric rv in Bernoulli process

- Memoryless


\eci
\end{frame}

%%%%%%%%%%%%%%%%%%%%%%%%%%%%%%%%%%%%%%%%%%%%%%%%%%%%%%
\begin{frame}{Poisson Process: Example}

\plitemsep 0.15in
\bci
\item Receive emails according to a Poisson process at rate $\lambda =5$ messages per hour

\item Mean and variance of mails received during a day

\onslide<2->{- 5* 24 = 120}

\item $\prob{\text{one new message in the next hour}}$

\onslide<3->{- $\cprob{1,1} = \frac{(5\cdot 1)^1 e^{-5 \cdot 1}}{1 !} = 5 e^{-5}$}

\item $\prob{\text{exactly two msgs during each of the next three hours}}$

\onslide<4->{- $\lf (\frac{5^2 e^{-5}}{2!} \ri)^3$}
\eci
\end{frame}

%%%%%%%%%%%%%%%%%%%%%%%%%%%%%%%%%%%%%%%%%%%%%%%%%%%%%%
\begin{frame}{Memoryless and Fresh-start Property}

\plitemsep 0.1in
\bci
\item \redf{Remind.} Similar property for Bernoulli processes, but here no time slots. 

\item<2-> \redf{Fresh-start at determinsitic time:} Start watching at time $t,$ then you see the Poisson process, independent of what has happened in the past. 

\item<3-> \redf{Fresh-start at random time:} Similarly holds. For example, when you start watching at random time $T_1$ (time of first arrival)

\item<4-> \redf{(Q3)} The $k$-th arrival time $Y_k$?

\item<5-> $k$-th inter-arrival time $T_k = Y_k = Y_{k-1},$ $k\ge 2,$ and $T_1 = Y_1.$

\item<6-> $Y_k = T_1 + T_2 + \cdots + T_k$ is sum of i.i.d. exponential rvs. 

\item<7-> $\expect{Y_k} = k/\lambda$ and $\var{Y_k} = k/\lambda^2$
\eci
\end{frame}

%%%%%%%%%%%%%%%%%%%%%%%%%%%%%%%%%%%%%%%%%%%%%%%%%%%%%%
\begin{frame}{PDF of $Y_k$}

\plitemsep 0.07in
\bci
\item<2-> For a given $\delta$, \redblank{3}{$\delta\cdot f_{Y_k}(y)$}: prob of $k$-th arrival over $[y,y+\delta].$

\item<4-> When $\delta$ is small, only one arrival occurs. Thus, 
 \aleq{
 \delta\cdot f_{Y_k}(y) &= \onslide<5->{\cbprob{\text{an arrival over $[y,y+\delta]$}} \times \cbprob{\text{$k-1$ arrivals before $y$}}} \cr
 &\onslide<6->{\approx \lambda \delta \times \cprob{k-1,y} = \lambda \delta \times \frac{\lambda^{k-1}y^{k-1}e^{-\lambda y}}{(k-1)!}}\cr
  \onslide<7->{f_{Y_k}(y) &= \frac{\lambda^k y^{k-1}e^{-\lambda y}}{(k-1)!}, \quad y\ge 0.}
}

\item<8-> This is called \redf{Erlang} rv. 

\bigskip
\item<9-> Time of first arrival: geometric / exponential
\item<9-> Time of $k$-th arrivals: Pascal / Erlang
\eci
\end{frame}





%%%%%%%%%%%%%%%%%%%%%%%%%%%%%%%%%%%%%%%%%%%%%%%%%%%%%%
\begin{frame}{Poisson Process vs. Bernoulli Process}

- $n= \tau/\delta$, $p = \lambda \delta$, $np = \lambda \tau$

\vspace{-0.8cm}
\begin{center}
\mypic{0.35}{L8_tau_delta.jpg}
\end{center}
\vspace{-0.8cm}
\centering
\mypic{0.65}{L8_poisson_bernoulli.png}
\end{frame}

%%%%%%%%%%%%%%%%%%%%%%%%%%%%%%%%%%%%%%%%%%%%%%%%%%%%%%
\begin{frame}{Example: Poisson Fishing (Problem 10, page 329)}

-  Catching fish: Poisson process $\lambda = 0.6$/hour.

- Fish for 2 hours. Within 2 hours, if he catches at least one fish, then he stops at 2 hours. Otherwise, he fishes until one fish is caught.

\medskip
\mytwocols{0.55}
{
\small
\plitemsep 0.05in
\bce[\bf (Q1)]
\item<2-> $\cprob{\text{fishing time $>$ 2 hours}}$ 

\smallskip
\onslide<3->{Method 1: $\cprob{0,2}$}

\onslide<4->{Method 2: $\cprob{T_1 >2}$}

\item<5->  $\cprob{\text{ 2 $<$ fishing time $<$ 5}}$ 

\smallskip
\onslide<6->{Method 1: $\cprob{0,2}(1-\cprob{0,3})$} 

\onslide<7->{Method 2:$\cprob{2 < T_1 <5}$}

\item<8-> $\cprob{\text{Catch at least two fish}}$

\smallskip
\onslide<9->{Method 1:$\sum_{k=2}^\infty = 1 - \cprob{0,2} - \cprob{1,2}$}

\onslide<10->{Method 2: $\cprob{Y_k \le 2}$}

\ece
}
{
\small
\plitemsep 0.05in
\bce[\bf (Q4)]
\item<11-> $\expect{\text{future fi. time}|\text{already fished for 3h}}$

\onslide<12->{Fresh-start. So, $\expect{exp(\lambda)} = 1/\lambda = 1/0.6$}

\item<13->[\bf (Q5)] $\expect{$F=$\text{total fishing time}}$
\aleq{
\onslide<14->{2 + \expect{F-2} &= 2+ \cprob{F=2}\cdot 0 + \cr
& \cprob{F >2}\cdot \expect{F-2 | F >2}\cr
&= 2 + \cprob{0,2}\cdot \frac{1}{\lambda}}
}
\item<15->[\bf (Q6)] $\expect{\text{number of fish}}$ = $\lambda \cdot 2 + \cprob{0,2}\cdot 1$
\ece

}

\end{frame}


%%%%%%%%%%%%%%%%%%%%%%%%%%%%%%%%%%%%%%%%%%%%%%%%%%%%%%
\begin{frame}{Roadmap}

\plitemsep 0.1in

\bci 
\item A lot of applications in engineering systems


\bigskip

\item \bluef{Basics on Random Process}

\medskip
\item \bluef{Bernoulli Process}
\item \bluef{Poisson Process}
\item \redf{Use of Bernoulli and Poisson Processes

- Coding of both processes

- Merge and Split
}

\medskip
\item Markov Chain

\eci 

\end{frame}


%%%%%%%%%%%%%%%%%%%%%%%%%%%%%%%%%%%%%%%%%%%%%%%%%%%%%%
\begin{frame}{Coding: Bernoulli Process and Poisson Process}

- Inter-arrival times facilitates coding of both processes

\smallskip
\centering
\redblank{1}{\mypic{0.8}{L8_coding_bernoulli.png}}

\smallskip
\redblank{1}{\mypic{0.8}{L8_coding_poisson.png}}


\end{frame}

%%%%%%%%%%%%%%%%%%%%%%%%%%%%%%%%%%%%%%%%%%%%%%%%%%%%%%
\begin{frame}{Sum of Independent Poisson rvs}

\plitemsep 0.1in

\bci 

\item $X \sim Poisson(\mu),$  $Y \sim Poisson(\nu),$  

\item<2-> \redf{(Q1)} $X \indep Y$?

\item<3-> \redf{(Q2)} Distribution of $X+Y$?

- Complex convolution, but any other easy way?

\item<4-> $X$ can be regarded as the number of arrivals of Poisson process with rate 1 over the time interval of length $\mu.$ 

\item<5-> Consecutive intervals of length $\mu$ and $\nu$

\item<6-> \redf{(Q1)} $X \indep Y$? \bluef{Yes}

\item<7-> \redf{(Q2)} Distribution of $X+Y$? \bluef{$Poisson(\mu + \nu)$}

\eci 


\end{frame}


%%%%%%%%%%%%%%%%%%%%%%%%%%%%%%%%%%%%%%%%%%%%%%%%%%%%%%
\begin{frame}{Split and Merge: Bernoulli Process}

\mytwocols{0.55}
{
\small
\plitemsep 0.05in
\bci
\item Split $Bernoulli(p)$ into two processes with biased coin of head probability $q$

\item<2-> Split decisions are independent of arrivals

\item<3-> Split processes: also Bernoulli processes

\item<3-> \redf{$Bernoulli(pq)$} and \redf{$Bernoulli(p(1-q))$}

\eci

\medskip
\centering
\mypic{0.9}{L8_split_bernoulli}

}
{
\small
\plitemsep 0.05in
\bci
\item Merge $Bernoulli(p)$ and $Bernoulli(q)$ into one.

\item<4-> Collided arrival is regarded just one arrival in the merged process

\item<5-> Merged process: \redf{$Bernoulli(1-(1-p)(1-q))$}
\eci

\centering
\mypic{0.9}{L8_merge_bernoulli}

}

\end{frame}

%%%%%%%%%%%%%%%%%%%%%%%%%%%%%%%%%%%%%%%%%%%%%%%%%%%%%%
\begin{frame}{Split and Merge: Poisson Process}

\plitemsep 0.1in
\bci
\item Split Poisson process $(\lambda)$ into two processes

\bci
\item<2-> Split based on the coin tossing with probability of head $p$

\item<3-> Poisson process $(p\lambda)$ and Poisson process $((1-p)\lambda)$
\eci

\item Merge from Poisson process $(\lambda_1)$ and Poisson process $(\lambda_2)$
\bci
\item<4-> Split based on the coin tossing with probability of head $p$

\item<5-> Poisson process ($\lambda_1 + \lambda_2$)

\item<6-> Bernoulli process of small interval $\delta$
\aleq{
\cprob{\text{0 arrivals in the merged process} } &\approx (1-\lambda_1 \delta)(1-\lambda_2 \delta) = 1-(\lambda_1 + \lambda_2)\delta + o(\delta)\cr
\cprob{\text{1 arrivals in the merged process} } & \approx \lambda_1 \delta (1-\lambda_2 \delta) + \lambda_2 \delta (1-\lambda_1 \delta) = (\lambda_1 + \lambda_2) \delta + o(\delta)
}
\eci
\eci
\end{frame}

%%%%%%%%%%%%%%%%%%%%%%%%%%%%%%%%%%%%%%%%%%%%%%%%%%%%%%
\begin{frame}{Competing Exponential}

\mytwocols{0.7}
{
\small
\plitemsep 0.1in
\bci
\item[1.] Two independent light bulbs have life times $T_a$ and $T_b$ of exponential distributions with $\lambda_a$ and $\lambda_b.$

\item<2-> \redf{(Q)} Distribution of $Z = \min\{T_a, T_b \}$?

\bigskip

\item<3-> $T_a$ and $T_b$ are the first arrival times of two Poisson processes of $\lambda_a$ and $\lambda_b.$

\item<5-> $Z$ is the first arrival time of merged Poisson process $(\lambda_a + \lambda_b)$. 

\item<6-> Thus, $Z \sim exp(\lambda_a + \lambda_b)$
\eci
}
{
\small
\plitemsep 0.05in
\bci
\item<7->[2.] Three independent light bulbs have life times $T$ of exponential distribution with $\lambda.$ 

\item<8-> \redf{(Q)} $\expect{\text{time until the last bulb burns out}}$?

\bigskip

\item<9-> Poisson process$(3\lambda)$ $\xrightarrow{\text{1st burn out}}$ Poisson process$(2\lambda)$ $\xrightarrow{\text{2st burn out}}$ Poisson process$(\lambda)$ 

\item<10-> $T_1$: time until the first burn-out, $T_2$: time until the second burn-out, $T_3$: time until the third burn-out

\item<11-> $T_1 \sim exp(3\lambda)$, $T_2 \sim exp(2\lambda)$, $T_3 \sim exp(\lambda)$
$$
\bexpect{T_1 + T_2 + T_3} = \frac{1}{3 \lambda} + \frac{1}{2 \lambda} + \frac{1}{ \lambda} $$

\eci
}

\end{frame}

%%%%%%%%%%%%%%%%%%%%%%%%%%%%%%%%%%%%%%%%%%%%%%%%%%%%%%
\begin{frame}{}
\vspace{2cm}
\LARGE Questions?

\end{frame}

%%%%%%%%%%%%%%%%%%%%%%%%%%%%%%%%%%%%%%%%%%%%%%%%%%%%%%
\begin{frame}{Review Questions}

\bce[1)]
\item 

\ece

\end{frame}




\end{document}
